\documentclass[a4paper,12pt]{scrartcl} 

%\usepackage[latin1]{inputenc} 
%Apple \usepackage[applemac]{inputenc} 
\usepackage[utf8]{inputenc}
\usepackage[ngerman]{babel}
\usepackage[T1]{fontenc}

%Das Paket erzeugt ein anklickbares Verzeichnis in der PDF-Datei.
\usepackage[hyperfootnotes=false,colorlinks=true,linkcolor=black,urlcolor=black]{hyperref}

%Das Paket wird fr die anderthalb-zeiligen Zeilenabstand bentigt
\usepackage{setspace}

%Einrückung eines neuen Absatzes
\setlength{\parindent}{0em}

%Definition der Rnder
\usepackage[paper=a4paper,left=30mm,right=30mm,top=30mm,bottom=30mm]{geometry} 

\usepackage{amsfonts}
\usepackage{amsmath}
\usepackage{cancel}
\usepackage{graphicx}
\usepackage{mathcomp}
\usepackage{polynom}
\usepackage{hyperref}

%Abstand der Fußnoten
\deffootnote{1em}{1em}{\textsuperscript{\thefootnotemark\ }}

%Regeln, bis zu welcher Tiefe (section,subsection,subsubsection) Überschriften angezeigt werden sollen (Anzeige der Überschriften im Verzeichnis / Anzeige der Nummerierung)
\setcounter{tocdepth}{3}
\setcounter{secnumdepth}{3}

%-------------------
%Ende des Kopfbereiches
%-------------------


\begin{document}

%Beginn der Titelseite
\begin{titlepage}
\begin{small}
\vfill {AKAD\\ 
Bachelor of Science (Wirtschaftsinformatik) \\ 
Modulzusammenfassung}
\end{small}


\begin{center}
\begin{Large}
\vfill {\textsf{\textbf{
WIM04 \\
\vspace*{1cm} 
Formelsammlung
}}}
\end{Large}
\end{center}

\begin{small}
\vfill Daniel Falkner \\ Rotbach 529 \\  94078 Freyung \\  daniel.falkner@akad.de \\ 
\today
\end{small}

\end{titlepage}
%Ende der Titelseite


%Inhaltsverzeichnis (aktualisiert sich erst nach dem zweiten Setzen)
\tableofcontents
\thispagestyle{empty}

%Beginn einer neuen Seite
\clearpage

%Anderthalbzeiliger Zeilenabstand ab hier
\onehalfspacing

\pagestyle{plain}


\section{Folgen}
Eine Serie von Zahlen oder Größen \\
5, 10, 4, 1 \\
$(a_n) = a_1, a_2, a_3, .., a_n$
\subsection{arithmetische Folgen}
\begin{itemize}
\item $a_{n+1} = a_n + d$ 
\item 7, 11, 15, 19, 23, 27, ... 
\item $\mapsto d = 4$ 
\end{itemize}

\subsubsection{Bildungsgesetz}
$a_n = a_1 + d * (n - 1)$

\subsection{geometrische Folgen}
\begin{itemize}
\item $a{n+1} = a_n * q$
\item 2, 6, 18, 54, 162, 486, ...
\item $\mapsto q = 3$
\end{itemize}

\subsubsection{Bildungsgesetz}
$a_n = a_1 * q^{n - 1}$
\\ \\
$q = \sqrt[n-1]{\cfrac{a_n}{a_1}}$


\section{Reihen}
Aus einer Folge ergibt sich eine Reihe \\
$(s_n) = s_1, s_2, s_3, ..., s_n$ \\
$(s_n) = a_1 + a_2 + a_3 + ... + a_n = \sum\limits_{j=1}^n a_j$

\subsection{arithmetische Reihen}
\begin{itemize}
\item $(a_n) = 7, 11, 15, 19, ... \mapsto a_1 = 7, d = 4$ 
\item $(s_n) = 7, 18, 33, 52, ...$
\end{itemize}

\subsubsection{Bildungsgesetz}
$s_n = \cfrac{n}{2} * (a_1 + a_n) = \cfrac{n}{2} * (2a_1 + (n - 1)d)$

\subsection{geometrische Reihen}
\begin{itemize}
\item $(a_n) = 2, 6, 18, 54, ... \mapsto a_1 = 2, q = 3$ 
\item $(s_n) = 2, 8, 26, 80, ...$
\end{itemize}

\subsubsection{Bildungsgesetz}
$s_n = a_1 * \cfrac{q^n - 1}{q - 1}, q \ne 1$

\section{Vollständige Induktion}
\begin{enumerate}
\item Zeigen das die Formeln für n = 1 gelten
\item Zeigen das die Formeln für n + 1 gelten
\begin{enumerate}
\item Induktionsannahme festhalten $a_n$ (zu beweisende Formel)
\item Die zubeweisende Formel für n + 1 herleiten $a_{n+1}$
\item Die Induktionsnahme + Ursprungsformel für n + 1 herleiten $a_{n+1}$
\end{enumerate}
\end{enumerate}

\section{Matrizen}
\subsection{Transponierte Matrix}
$A^T$ entsteht durch Vertauschen der Zeilen mit den Spalten von $A$\\

Beispiel: \\
$A_{(2,3)} = 
\begin{bmatrix}
 1 & 2 & 3\\
 -2 & 4 & -1 \\
\end{bmatrix}
A_{(3,2)}^T = 
\begin{bmatrix}
 1 & -2\\
 2 & 4\\
 3 & -1
\end{bmatrix}
$
\subsection{Addition}
\subsubsection{vom selben Typ}
die gleichstehenden Elemente addieren und zu einer neuen Matrix zusammenfassen

\subsection{Multiplikation}
\subsubsection{mit einer reellen Zahl (Skalar)}
alle Elemente der Matrix mit der Zahl multiplizeren

\subsubsection{zweier Matrizen}
Zwei Matrizen sind multiplikationsverträglich wenn die Spaltenanzahl von A mit der Zeilanzahl von B übereinstimmt. 
Eine Hilfe bietet das Falk-Schema \footnote{\url{http://de.wikipedia.org/wiki/Falksches_Schema}}

\subsubsection{spezielle Matrixprodukte}
\begin{itemize}
\item Zeilenvektor * Spaltenvektor = Skalar
\item Spaltenvektor * Zeilenvekor = Matrix
\end{itemize}

\subsection{Inverse}
A vom Typ (n,n) ist regulär, d.h. $A^{-1}$ (Inverse Matrix) existiert. Dann ist die Matrixgleichung A * X = B eindeutig lösbar.
\subsubsection{Bestimmtung der inversen Matrix}
\begin{itemize}
\item Die Inverse $A^{-1}$ lässt sich mit dem Gauß-Jordan-Verfahren ermitteln \footnote{\url{http://de.wikipedia.org/wiki/Gau\%DF-Jordan-Algorithmus}}
\item Eine quadratische Matrix A ist genau dann invertierbar, wenn ihre Determinate |A| ungleich Null ist
\end{itemize}


\end{document}

